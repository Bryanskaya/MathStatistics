\section{Формулы для вычисления величин}
Выборка:  $\vec{x} = (x_1, ..., x_n) $

\subsection{Максимальное значение выборки}
\begin{equation}
	M_{max} = max\{x_1, ..., x_n\}
\end{equation}

\subsection{Минимальное значение выборки}
\begin{equation}
	M_{min} = min\{x_1, ..., x_n\}
\end{equation}

\subsection{Размах выборки}
\begin{equation}
	R = M_{max} - M_{min}
\end{equation}

\subsection{Выборочное среднее}
\begin{equation}
	\hat\mu(\vec x_n) = \frac 1n \sum_{i=1}^n x_i
\end{equation}

\subsection{Несмещённая оценка дисперсии (состоятельная оценка)}
\begin{equation}
	S^2(\vec x_n) = \frac 1{n-1} \sum_{i=1}^n (x_i-\overline x_n)^2
\end{equation}


\section{Эмпирическая плотность и гистограмма}
\textbf{Интервальным статистическим рядом} называют таблицу:  \\  

\begin{tabular}{ | l | l | l | l | l | }
	\hline
	$J_1$ & ... & $J_i$ & ... & $J_m$ \\ 
	\hline
	$n_1$ & ... & $n_i$ & ... & $n_m$ \\
	\hline
\end{tabular}\\

здесь $n_i$ - число элементов выборки $\vec x$, которые попали в $J_i$.  \\

Пусть для выборки $\vec x$ построен интервальный статистический ряд, \textbf{эмпирической плотностью} называется функция:  
\begin{equation}
	\hat{f}(x) = \begin{cases}
		\frac{n_{i}}{n\Delta}, & x\in J_{i};\\
		0,                     & x\notin J.
	\end{cases}
\end{equation}

График импирической плотности называется \textbf{гистограммой}.

\section{Эмпирическая функция распределения}
\textbf{Эмпирической функцией распределения}, отвечающей выборке $\vec x$, называют функцию:
\begin{equation}
	\hat{F}(x) = \frac{n(x, \vec{x})}{n},
\end{equation}
	где $n(x, \vec{x})$ - число элементов вектора $\vec x$, которые имеют значение меньше, чем $x$; $n$ — объём выборки.