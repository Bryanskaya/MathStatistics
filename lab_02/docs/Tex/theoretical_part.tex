\section{Формулы для вычисления величин}
\subsection{Определение $\gamma$-доверительного интервала}
Доверительным интервалом уровня $\gamma$ для параметра $\theta$ называется интервал ($\underline\theta(\vec X)$, $\overline\theta(\vec X)$), образующий выборочными значениями статистики $\underline\theta$ и $\overline\theta$.\\
\begin{equation*}
	P\{\underline\theta(\vec X) <= x <= \overline\theta(\vec X) \} = \gamma
\end{equation*}

\section{Границы $\gamma$-доверительного интервала}

Пусть $\vec x = (x_1, ..., x_n)$ — случайная выборка объема $n$ из генеральной совокупности $X$, распределенной по нормальному закону с параметрами $\mu$ и $\sigma^{2}$.

\textbf{$\gamma$-доверительный интервал для математического ожидания}
\begin{equation*}
	P\{\overline{X} - \dfrac{S(\vec{X})}{\sqrt{n}} t^{St(n-1)}_{\frac{1+\gamma}{2}} <= \mu <= \overline{X} + \dfrac{S(\vec{X})}{\sqrt{n}} t^{St(n-1)}_{\frac{1+\gamma}{2}} \} = \gamma
\end{equation*}
Т.е.\\
\begin{equation*}
	\underline\mu(\vec X) = \overline{X} - \dfrac{S(\vec{X})}{\sqrt{n}} t^{St(n-1)}_{\frac{1+\gamma}{2}}
\end{equation*}

\begin{equation*}
	\overline\mu(\vec X)  = \overline{X} + \dfrac{S(\vec{X})}{\sqrt{n}} t^{St(n-1)}_{\frac{1+\gamma}{2}}
\end{equation*}

\subsection{Оценка для математического ожидания}

\textbf{$\gamma$-доверительный интервал для дисперсии}
\begin{equation*}
	P\{ \dfrac{S^2(\vec{X}) (n-1)}{t^{\chi^{2}(n-1)}_{\frac{1+\gamma}{2}}} <= \sigma^2 <= \dfrac{S^2(\vec{X}) (n-1)}{t^{\chi^{2}(n-1)}_{\frac{1+\gamma}{2}}}  \} = \gamma
\end{equation*}
Т.е.\\
\begin{equation*}
	\underline\sigma(\vec X) = \dfrac{S^2(\vec{X}) (n-1)}{t^{\chi^{2}(n-1)}_{\frac{1+\gamma}{2}}}
\end{equation*}

\begin{equation*}
	\overline\sigma(\vec X)  = \sigma^2 <= \dfrac{S^2(\vec{X}) (n-1)}{t^{\chi^{2}(n-1)}_{\frac{1+\gamma}{2}}}
\end{equation*}






