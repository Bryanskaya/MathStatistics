\textbf{Цель работы:} построение доверительных интервалов для математического ожидания и дисперсии нормальной случайной величины. \\

\qquad\textbf{Содержание работы:}

\begin{enumerate}
	\item Для выборки объема $n$ из нормальной генеральной совокупности $X$ реализовать в виде программы на ЭВМ
	\begin{itemize}
		\item[а)] вычисление точечных оценок $\hat\mu(\vec x_n)$ и $S^2(\vec x_n)$ математического ожидания $MX$ и дисперсии $DX$ соответственно;
		\item[б)] вычисление нижней и верхней границ $\underline\mu(\vec x_n)$, $\overline\mu(\vec x_n)$ для $\gamma$-доверительного интервала для математического ожидания $MX$;
		\item[в)] вычисление нижней и верхней границ $\underline\sigma^2(\vec x_n)$, $\overline\sigma^2(\vec x_n)$ для $\gamma$-доверительного интервала для дисперсии $DX$;
	\end{itemize}
	\item вычислить $\hat\mu$ и $S^2$ для выборки из индивидуального варианта;
	\item для заданного пользователем уровня доверия $\gamma$ и $N$ – объёма выборки из индивидуального варианта:
	\begin{itemize}
		\item[а)] на координатной плоскости $Oyn$ построить прямую $y = \hat\mu(\vec{x_N})$, также графики функций $y = \hat\mu(\vec x_n)$, $y = \underline\mu(\vec x_n)$ и $y = \overline\mu(\vec x_n)$ как функций объема $n$ выборки, где $n$ изменяется от 1 до $N$;
		\item[б)] на другой координатной плоскости $Ozn$ построить прямую $z = S^2(\vec{x_N})$, также графики функций $z = S^2(\vec x_n)$, $z = \underline\sigma^2(\vec x_n)$ и $z = \overline\sigma^2(\vec x_n)$ как функций объема $n$ выборки, где $n$ изменяется от 1 до $N$.
	\end{itemize}
\end{enumerate}
